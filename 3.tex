%!TEX root = thesis.tex"`

\chapter{Программная реализация метода функционала плотности для модели ВЭГ}

\section{Метод функционала электронной плотности}
Метод функционала плотности (\acrshort{mfp}) представляет собой точную\\ квантовомеханическую теорию для системы взаимодействующих квантовых частиц во внешнем потенциале $V_{ext} (\vec{r})$.
Сам метод основан на двух строго доказанных теоремах \cite{hohenberg:dft}:
\begin{enumerate}
    \item Для любой системы взаимодействующих частиц во внешнем потенциале $V_{ext} (\vec{r})$, потенциал $V_{ext} (\vec{r})$ с точностью до произвольной постоянной определяется электронной плотностью основного состояния $n_0 (\vec{r})$.
    \item Энергия невырожденного основного состояния системы для любого внешнего потенциала $V_{ext} (\vec{r})$ является функционалом электронной плотности $E[n(\vec{r})]$. Основное состояние системы является минимумом данного функционала, который достигается при плотности, соответствующей основному состоянию системы $n_0 (\vec{r})$.
\end{enumerate}

Из первой теоремы следует, что гамильтониан системы с точностью до аддитивной константы определяется электронной плотностью основного состояния $n_0 (\vec{r})$.
Следовательно, определены многочастичные волновые функции для всех состояний (основного и возбужденных).
Таким образом, все свойства системы полностью определены, если известна электронная плотность основного состояния $n_0 (\vec{r})$.
Из второй теоремы следует, что при известном функционале $E[n]$ можно определить плотность и энергию основного состояния.

Функционал энергии в формулировке \cite{hohenberg:dft} можно записать следующим образом:
\begin{equation}
    \label{eq:dft:E_HK-def}
    E_{HK} [n] = T[n] + E_{int} [n] + \int\limits_{}^{} d^3 r V_{ext} (\vec{r}) n(\vec{r}) + E_{II} 
    \equiv F_{HK} [n] + \int\limits_{}^{}d^3 r V_{ext} (\vec{r}) n(\vec{r}) + E_{II},
\end{equation}
где $E_{II} $ ---~ энергия взаимодействия ядер. Функционал $F_{HK} [n]$ включает в себя кинетическую и потенциальную энергию системы взаимодействующих электронов,
\begin{equation}
    \label{eq:dft:F_HK-def}
    F_{HK} [n] = T[n] + E_{i n t} [n]. 
\end{equation}
Минимум функционала \eqref{eq:dft:E_HK-def} опрелеляет основное состояние жнергии системы и ее электронную плотность.

Теория функционала плотности допускает обобщение на случай конечных температур \cite{mermin:dft-temperatures}.
Функционал энергии от электронной плотности заменяется на функционал большого термодинамического потенциала $\Omega$ от оператора плотности $\hat{\rho}$:
\begin{equation}
    \label{eq:dft:Omega_rho}
    \Omega [\hat{\rho}] = \mathrm{Sp} \left[ \hat{\rho} (\hat{H} - \mu \hat{N}) + \frac{1}{\beta}\, \ln \hat{\rho} \right]. 
\end{equation}
Минимум этого функционала совпадает с термодинамически равновесным выражением для большого термодинамического потенциала:
\begin{equation}
    \label{eq:dft:Omega-min}
    \Omega = \Omega [\hat{\rho_0}] = -\ln \mathrm{Sp}\, e^{-\beta (\hat{H} - \mu \hat{N})}, 
\end{equation}
где $\hat{\rho}_0$ ---~ оператор плотности в большом каноническом ансамбле (\acrshort{bka}),
\begin{equation}
    \label{eq:dft:rho_0-bka}
    \hat{\rho}_0 = \frac{e^{-\beta (\hat{H} - \mu \hat{N})} }{\mathrm{Sp}\, e^{-\beta (\hat{H} - \mu \hat{N})} }\, .
\end{equation}

Теормера Мермина \cite{mermin:dft-temperatures} утверджает, что не только энергия, но также и все термодинамические функции (энтропия, теплоемкость) являются функционалами равновесной плотности.

Одна из основных проблем \acrshort{mfp} ---~ это отсутствие в общем случае прямой связи между кинетической энергией и функцией электронной плотности.
Для решения этой проблемы в \cite{kohn-sham:dft-kinetic-approximation} высказано предположение, что основное состояние взаимодействующей системы частиц совпадает с основным состоянием эквивалентной системы невзаимодействующих частиц, а само взаимодействие учитывается с помощью так называемого <<обменно-корреляционного>> функционала, зависящего от электронной плотности.
Результаты такого приема оказались весьма хорошими и на данный момент все существующие методы расчета на основе \acrshort{mfp} используют это приближение.

Функциоанл энергии $E[n]$ в описываемом приближении имеет вид:
\begin{equation}
    \label{eq:dft:E_KS-def}
    E_{KS} = T_{s} [n] + \int\limits_{}^{} d \vec{r} V_{ext} (\vec{r}) n(\vec{r}) + E_{Hartree} [n] + E_{II}  + E_{xc} [n],  
\end{equation}
где электронная плотность определяется выражением
\begin{equation}
    \label{eq:dft:density-kohn}
    n (\vec{r}) = \sum\limits_{\sigma}^{} n(\vec{r}, \sigma) = \sum\limits_{\sigma}^{} \sum\limits_{i=1}^{N^{\sigma} } |\psi_i^{\sigma} (\vec{r})|^2, 
\end{equation}
кинетическая энергия в приближении невзаимодействующих частиц:
% \begin{equation}
%     \label{eq:dft:T_s-def}
%     T_{s}  = - \frac{1}{2}\, \sum\limits_{\sigma}^{}\sum\limits_{i=1}^{N^{\sigma}} \bra{\psi_i^{\sigma} } \nabla^2 \ket{\psi_i^{\sigma}} 
%      = \frac{1}{2}\, \sum\limits_{\sigma}^{} \sum\limits_{i=1}^{N^{\sigma} } |\nabla \psi_i^{\sigma} |^2,
% \end{equation}
% а энергия кулоновского взаимодействия электронов
% \begin{equation}
%     \label{eq:dft:E_Hartree-def}
%     E_{Hartree} [n] = \frac{1}{2}\, \int\limits_{}^{}d^3 r d^3 r' \frac{n(\vec{r}) n(\vec{r}')}{|\vec{r} - \vec{r}'}.
% \end{equation}

\section{Метод функционала плотности для модели ВЭГ}

\section{Алгоритм моделирования ВЭГ методом функционала плотности при конечной температуре}
