%!TEX root = thesis.tex"`

\chapter{Модель взаимодействующего электронного газа на однородном компенсирующем фоне}

\section{Классическая модель однокомпонентной плазмы}
Классическая модель однокомпонентной плазмы (\acrshort{okp}) ~--- это модель точечных ионов, помещенных для обеспечения электронейтральности в равномерно распределенную среду заряда противоположного знака.
% TODO: Не уверен, стоит ли оставлять примеры
Такая модель является хорошим приближением для плазмы сверхвысоких давлений, реализующихся, например, в центре белых карликов и тяжелых планет типа Юпитера.
В этих случаях под действием давления вещество полностью ионизовано, а вырожденные электроны обладают достаточной кинетической энергией $\epsilon_k \approx \epsilon_F$, чтобы образовать почти однородное фоновое распределение зарядовой плотности.
Модель \acrshort{okp} является простейшей нетривиальной моделью плазмы, так как вид потенциала взаимодействия здесь не вызывает сомнений, а отсутствие квантовых эффектов позволяет исключить из рассмотрения образование связанных состояний (молекул, атомов, ионов) и влияние вырождения и интерференции. Важным свойством модели \acrshort{okp} является зависимость ее свойств только от одного параметра $\gamma = q_i^2/(kT\bar{r})$ в силу однородности кулоновского потенциала, где $q_i$~--- заряд иона, $\bar{r}$~--- среднее расстояние между ионами, обычно определяемое соотношением $4\pi\bar{r}^3/3 = 1/n_i$, $n_i$~--- концентрация ионов.  Более подробно с моделью и ее основными результатами можно ознакомиться в \cite{Fortov:neid_plasma}.

Большое количество результатов по \acrshort{okp} получено методом Монте-Карло.
Основным параметром в этих результатах является бинарная корреляционная функция $g(r)$.
На ее основании возможно вычислить внутреннюю энергию:
\begin{equation}
    \label{eq:internal_energy_g}
    u=3 n_{\mathrm{i}} k T / 2+u_{\text{кор}},
\end{equation}
\begin{equation}
    \label{eq:u_corr_g}
    u_{\text{кор}} /\left(n_{\mathrm{i}} k T\right)=\left(n_{\mathrm{i}} / 2 k T\right) \int d \mathbf{r}\left(Z^{2} e^{2} / r\right)[g(r)-1].
\end{equation}

Результаты вычислений методом Монте-Карло для $1 \leq \gamma \leq 160$ были аппроксимированы в \cite{Slattery:u_corr_approximation} с погрешностью $3 \cdot 10^{-5}$ следующим выражением:
\begin{equation}
    \label{eq:u_corr_approx}
    u_{\text{кор}} /\left(n_{\mathrm{i}} k T\right)=a \gamma+b \gamma^{1 / 4}+c \gamma^{-1 / 4}+d
\end{equation}
где $a=-0,89752$, $b=0,94544$, $c=0,17954$, $d=-0,80049$.

Давление можно получить по теореме вириала $p V = u/3$.
\begin{equation}
    \label{eq:p_approx}
    p = \cfrac{u}{3V}
    = \cfrac{n_\mathrm{i} kT}{3V} \left( \frac{3}{2}
    + a \gamma + b \gamma^{1/4} + c \gamma^{-1/4} + d \right)
\end{equation}
Интегрируя \eqref{eq:u_corr_approx} по $\gamma$, можно получить свободную энергию \acrshort{okp}:
\begin{multline}
    \label{eq:f-density-gas-approx}
    f=F /\left(n_{\mathrm{i}} k T\right) = \\
    =a \gamma+4\left(b \gamma^{1 / 4}-c \gamma^{-1 / 4}\right)+(d+3) \ln \gamma-(a+4 b-4 c+1,135)
\end{multline}

Заметим, что в соответствии с \eqref{eq:p_approx} высокое значение $\gamma$ приводит к отрицательному давлению компонента (поскольку $a < 0$).
Это не приводит к неустойчивости \acrshort{okp}: полное давление оказывается положительным за счет большого давления электронного газа.

Для \acrshort{okp} возможно явление кристаллизации, впервые рассмотренное Вигнером в \cite{Wigner:plasma_condensation}.
Было показано, что классический электронный газ с концентрацией $n_e$ на фоне компенсирующего заряда должен при достаточно низких концентрациях образовывать упорядоченную структуру.
Это вызвано тем, что стабилизирующая решетку кулоновская энергия $V_{\mathrm{C}} \sim e^{2} n_{\mathrm{e}}^{1 / 3} \sim e^{2} / \bar{r}$ при расширении плазмы уменьшается медленнее, чем разрушающая решетку кинетическая энергия $\varepsilon_\text{к} \sim \varepsilon_{F} \sim \hbar^{2} n_{\mathrm{e}}^{2 / 3} / 2 m$.
Поэтому при малых плотностях кинетическая энергия $\varepsilon_\text{к} \sim n_{\mathrm{e}}^{2 / 3}$ становится меньше потенциальной $V_{\mathrm{C}} \sim n_{\mathrm{e}}^{1 / 3}$ и не способна разрушить упорядоченную структуру электронов, возникшую из-за отталкивания.

В работе \cite{Slattery:u_corr_approximation} были получены результаты вычислений для кристаллической \acrshort{okp} с объемно-центрированной решеткой. Избыточная энергия при $160~\lesssim~\gamma~\lesssim~300$ в \cite{Slattery:u_corr_approximation} аппроксимирована выражением:
\begin{equation}
    \label{eq:u_corr_solid_approx}
    u_{\text{кор}} / n_{\mathrm{i}} k T=a_{\mathrm{BCC}} \gamma+\frac{3}{2}\, b \gamma^{-2}=0,895929 \gamma+1,5+2980 / \gamma^{2}
\end{equation}
Из выражения \eqref{eq:u_corr_solid_approx} следует выражение для плотности свободной энергии решетки:
\begin{equation}
    \label{eq:f-density-solid-approx}
    f(\gamma)=-0,895929+9 \gamma / 2-1,8856-1490 / \gamma^{2}.
\end{equation}
Зависимости свободных энергий газовой \eqref{eq:f-density-gas-approx} и твердой \eqref{eq:f-density-solid-approx} фаз пересекаются при $\gamma_m = 165$ \cite{Slattery:u_corr_approximation}.
При этом значении параметра неидеальности происходит вигнеровская кристаллизация \acrshort{okp}.

\section{Суммирование Эвальда}
\label{sec:ewald}
Для расчета энергии в системах с дальнодействующими потенциалами широко используется метод суммирования по Эвальду \cite{ewald:summation_original}.
В данном методе потенциал взаимодействия разбивается на два слагаемых: короткодействующее и дальнодействующее.
Первое слагаемое вычисляется в действительном пространстве, в то время как для расчета второго используется преобразование Фурье.
По сравнению с прямым подсчетом данный метод дает быструю сходимость по энергии, что обеспечивает высокую точность и скорость расчетов.

Более подробно, в методе суммирования по Эвальду предлагается переписать потенциал взаимодействия $\phi(\vec{r})$ в виде:
\begin{equation}
    \label{eq:ewald:potention_def}
    \phi(\vec{r}) \eqdef \phi_{sr}(\vec{r})+\phi_{lr}(\vec{r}),
\end{equation}
где $\phi_{sr}(\vec{r})$ отвечает за короткодействующую часть, энергия которой быстро сходится, а $\phi_{lr}(\vec{r})$ соответствует дальнодействующей части, энергия которой суммируется в Фурье-пространстве.

Предполагается, что энергия короткодействующей части потенциала суммируется непосредственно и основная проблема состоит в суммировании дальнодействующего вклада в потенциал.
Также из-за использования Фурье-преобразования метод неявно подразумевает, что рассматриваемая система является бесконечно периодической, что является разумным предположением для кристаллических структур.

Будем называть примитивной ячейкой минимальную повторяющуюся часть периодической системы.
Также для определенности выберем одну ячейку и назовем ее \textit{центральной}, а остальные будем называть \textit{образами}.

Энергия дальнодействующего взаимодействия есть сумма энергий взаимодействия между зарядами центральной ячейки и всех остальных зарядов решетки.
Ее можно записать в виде:
\begin{equation}
    \label{eq:ewald:energy_lr}
    E_{lr}=\iint d \vec{r} d \vec{r'} \rho_{tot}(\vec{r}) \rho_{u c}\left(\vec{r'}\right) \phi_{lr}\left(\vec{r}-\vec{r'}\right),
\end{equation}
где введена плотность заряда примитивной ячейки $\rho_{uc} (\vec{r})$, представляющая из себя сумму по зарядам внутри \textit{одной} ячейки:
\begin{equation}
    \label{eq:ewald:rho_uc_def}
    \rho_{uc}(\vec{r}) \eqdef \sum_{\text{charges}\ k} q_{k} \delta\left(\vec{r}-\vec{r}_{k}\right)
\end{equation}
и суммарная плотность заряда $\rho_{tot}(\vec{r})$, представляющая из себя ту же сумму, но с добавлением зарядов от образов:
\begin{equation}
    \label{eq:ewald:rho_tot_def}
    \rho_{tot}(\vec{r}) \eqdef \sum_{n_{1}, n_{2}, n_{3}} \sum_{\text{charges}\,k} q_{k} \delta\left(\vec{r}-\vec{r}_{k}-n_{1} \vec{a}_{1}-n_{2} \vec{a}_{2}-n_{3} \vec{a}_{3}\right).
\end{equation}
Здесь $\delta (\vec{x})$ ---~ дельта-функция Дирака, $\vec{a_1}$, $\vec{a_2}$, $\vec{a_3}$ ---~ вектора решетки и $n_1$, $n_2$, $n_3$ пробегают по всем целым числам.

На \eqref{eq:ewald:rho_tot_def} можно смотреть как на свертку \eqref{eq:ewald:rho_uc_def} с \textit{функцией ячейки} $L (\vec{r})$:
\begin{equation}
    \label{eq:ewald:lattice_function_def}
    L(\vec{r}) \eqdef \sum_{n_{1}, n_{2}, n_{3}} \delta\left(\vec{r}-n_{1} \vec{a}_{1}-n_{2} \vec{a}_{2}-n_{3} \vec{a}_{3}\right).
\end{equation}
Для полученной свертки имеем:
\begin{equation}
    \label{eq:ewald:rho_tot_conv_prod}
    \tilde{\rho}_{tot}(\vec{k}) = \tilde{L}(\vec{k}) \tilde{\rho}_{u c}(\vec{k}),
\end{equation}
где тильдами обозначены Фурье-образы.

Для функции ячейки $L(\vec{r})$ Фурье-образ есть:
\begin{equation}
    \label{eq:ewald:lattice_function_image}
    \tilde{L}(\vec{k})=\frac{(2 \pi)^{3}}{\Omega} \sum_{m_{1}, m_{2}, m_{3}} \delta\left(\vec{k}-m_{1} \vec{b}_{1}-m_{2} \vec{b}_{2}-m_{3} \vec{b}_{3}\right),
\end{equation}
где $\vec{b_1} = \cfrac{\vec{a}_{2} \times \vec{a}_{3}}{\Omega}$, $\vec{b_2}$ и $\vec{b_3}$ получаются циклическими перестановками, индексы $m_1$, $m_2$, $m_3$ по-прежнему пробегают все целые числа и $\Omega$ есть объем центральной ячейки: $\Omega~=~\vec{a_1} \cdot (\vec{a_2} \times \vec{a_3})$.
Стоит заметить, что и $L (\vec{r})$, и $\tilde{L} (\vec{r})$ являются действительными четными функциями.

Вводя эффективный одночастичный потенциал
\begin{equation}
    \label{eq:ewald:v_def}
    v(\vec{r}) \eqdef \int d \vec{r'} \rho_{u c}\left(\vec{r'}\right) \phi_{lr}\left(\vec{r}-\vec{r'}\right),
\end{equation}
можно переписать \eqref{eq:ewald:energy_lr} в виде:
\begin{equation}
    \label{eq:ewald:energy_v_rewrite}
    E_{lr}=\int d \vec{r} \rho_{tot}(\vec{r}) v(\vec{r})
\end{equation}
Заметим, что \eqref{eq:ewald:v_def} тоже является сверткой, что позволяет записать:
\begin{equation}
    \label{eq:ewald:v_fourier_conv_prod}
    \tilde{V}(\vec{k}) \eqdef \tilde{\rho}_{uc}(\vec{k}) \tilde{\Phi}(\vec{k}),
\end{equation}
где, как и в \eqref{eq:ewald:rho_tot_conv_prod}, тильдами обозначены Фурье-образы и введен Фурье-образ \eqref{eq:ewald:v_def}:
\begin{equation}
    \label{eq:ewald:v_fourier}
    \tilde{V}(\vec{k})=\int d \vec{r} v(\vec{r}) e^{-i \vec{k} \cdot \vec{r}}.
\end{equation}

Согласно теореме Планшереля, суммирование для получения $E_{lr}$ можно провести в Фурье-пространстве, что дает:
\begin{multline*}
    E_{l r} = \int \frac{d \vec{k}}{(2 \pi)^{3}} \tilde{\rho}_{tot}^{*}(\vec{k}) \tilde{V}(\vec{k})
    = \int \frac{d \vec{k}}{(2 \pi)^{3}} \tilde{L}^{*}(\vec{k})\left|\tilde{\rho}_{u c}(\vec{k})\right|^{2} \tilde{\Phi}(\vec{k}) = \\
    = \frac{1}{\Omega} \sum_{m_{1}, m_{2}, m_{3}}\left|\tilde{\rho}_{u c}(\vec{k})\right|^{2} \tilde{\Phi}(\vec{k}),
\end{multline*}
\begin{equation}
    \label{eq:ewald:energy_final}
    E_{lr} = \frac{1}{\Omega} \sum_{m_{1}, m_{2}, m_{3}}\left|\tilde{\rho}_{u c}(\vec{k})\right|^{2} \tilde{\Phi}(\vec{k}),
\end{equation}
где в суммировании $\vec{k}=m_{1} \vec{b}_{1}+m_{2} \vec{b}_{2}+m_{3} \vec{b}_{3}$.

Формула \eqref{eq:ewald:energy_final} есть основной результат.
Фурье-образ $\tilde{\rho}_{uc} (\vec{k})$ можно подсчитать, к примеру, алгоритмами \acrshort{bpf} (\textit{англ.} \acrshort{fft}).
Если найден фурье-образ $\tilde{\rho}_{uc} (\vec{k})$, то суммирование (или интегрирование) по $\vec{k}$ становится простым и приобретает быструю сходимость.
Стоит отметить, что используемая примитивная ячейка должна быть электронейтральной во избежание бесконечных сумм.

На практике суммирование \eqref{eq:ewald:energy_final} ведется в конечном диапазоне значений $m_1$, $m_2$, $m_3$.
Благодаря особенностям метода отбрасывание остальных членов приводит к небольшой потере точности, однако значительно уменьшает вычислительное время.
Оценку возникающих из-за отбрасывания ошибок можно найти, например, в \cite{ewald:errors_study}.

\section{Гамильтониан модели \texorpdfstring{\acrshort{veg}}{ВЭГ}}
Взаимодействующий электронный газ (\acrshort{veg}) ---~ это физическая модель электронного газа, в которой учитывается электростатическое взаимодействие между всеми зарядами в системе.
В данной работе рассматривается важный частный случай этой модели~--- так называемая модель <<желе>>, в которой электронейтральность обеспечивается компенсирующим фоном положительного заряда с постоянной полностью (так, чтобы суммарный заряд системы был равен нулю). Фон предполагается однородным и несжимаемым.

\textit{При изложении данной главы будем использовать систему единиц СГС.}

Для модели <<желе>> из $N$ электронов, заключенных в объеме $V$, имеющих концентрацию $\rho (\vec{r}) = \sum_{i=1}^{N} \delta(\vec{r}-\vec{r_i})$ при концентрации фоновых зарядов $n(\vec{R}) = N/V$, гамильтониан имеет вид \cite{jel:mahan_hamiltonian}:

\begin{equation}
    \label{eq:jel:ham_four_sum}
    \hat{H}=\hat{H}_{e l}+\hat{H}_{b a c k}+\hat{H}_{e l-b a c k}.
\end{equation}
Здесь $\hat{H}_{e l}$ ---~ это электронный гамильтониан, состоящий из кинетической энергии и электрон-электронного отталкивания:
\begin{equation}
    \label{eq:jel:ham_el_def}
    \hat{H}_{e l}=\sum_{i=1}^{N} \frac{\hat{\vec{p}}_i^2}{2 m}+\sum_{i<j}^{N} \frac{e^{2}}{\left|\hat{\vec{r}}_{i}-\hat{\vec{r}}_{j}\right|}.
\end{equation}
%
$\hat{H}_{b a c k}$ есть гамильтониан положительного фонового заряда, описывающий его взаимодействие с самим собой. Усредненное значение $\hat{H}_{b a c k}$ есть:
\begin{equation}
    \label{eq:jel:ham_back_expect}
    \left\langle\hat{H}_{b a c k}\right\rangle
    = \frac{e^{2}}{2} \int_{V} d \vec{R} \int_{V} d \vec{R}' \frac{n(\vec{R}) n\left(\vec{R}'\right)}{\left|\vec{R}-\vec{R}'\right|}
    = \frac{N^{2}}{2 V} \lim _{\mathbf{q} \rightarrow 0} v_{q},
\end{equation}
где введена фурье-компонента кулоновского потенциала $v_{q}=\frac{4 \pi e^{2}}{q^{2}}$.
%
Вклад электростатического взаимодействия электрона с фоном описывается членом $H_{el-back}$:
\begin{multline}
    \label{eq:jel:ham_el-back_expect}
    \left\langle\hat{H}_{e l-b a c k}\right\rangle
    = -\int_{V} d \vec{r} \int_{V} d \vec{R}\, \frac{e^{2} \rho(\vec{r}) n(\vec{R})}{|\vec{r}-\vec{R}|} = {} \\
    = -\frac{N^{2}}{V} \lim _{\mathbf{q} \rightarrow 0} v_{q}
    = -2 \left\langle\hat{H}_{b a c k}\right\rangle.
\end{multline}

Для конечных систем результаты \eqref{eq:jel:ham_back_expect} и \eqref{eq:jel:ham_el-back_expect} получаются конечными, поскольку $q$ достигает отличное от нуля минимальное значение, зависящее от объема системы.

Для систем с периодическими граничными условиями суммирование кулоновского взаимодействия можно провести упомянутым выше методом Эвальда (см. раздел \ref{sec:ewald}).
Учтем также возможное размытие заряда относительно его равновесного положения в форме функции Гаусса \cite{jel:blinov}.

Введем потенциал в точке $\vec{r}$, создаваемый всей решеткой без одного лишь заряда $j$, но с учетом его образов:

\begin{equation}
    \label{eq:jel:phi_no_j}
    \phi_{-j}(\vec{r})=\frac{1}{\varepsilon} \sum_{\vec{n}} \sum_{i=1}^{N} {}^\prime \frac{q_{i}}{\left|\vec{r}-\vec{r}_{i}+\vec{n} L\right|},
\end{equation}
где штрих у суммы обозначает, что $i \neq j$ при $\vec{n} = 0$.
Здесь $q_i$ указан для того, чтобы получающиеся результаты не были привязаны к типам и значениям зарядов.
Используя выражения для полной энергии:
\begin{equation}
    \label{eq:jel:coloumb_energy}
    E = \frac{1}{2} \sum_{j=1}^{N} q_{j} \phi_{-j}\left(\vec{r}_j\right)
\end{equation}
и плотности заряда:
\begin{equation}
    \rho_{i}(\vec{r})=q_{i} \cdot \delta\left(\vec{r}-\vec{r}_{\mathrm{i}}\right),
\end{equation}
получим полную энергию в виде:
\begin{equation}
    \label{eq:jel:ewald_E_coloumb_total}
    E = \frac{1}{\epsilon} \sum_{\vec{n}} \sum_{i=1}^{N} \sum_{j=1}^{N} {}^\prime \iint \frac{\rho_{i}(\vec{r}) \cdot
    \rho_{j}\left(\vec{r}'\right)}{\left|\vec{r}-\vec{r}^{\prime}+\vec{n} L\right|} d^3 \vec{r}\, d^3 \vec{r}^{\prime}.
\end{equation}

Разбиваем заряды на части:
\begin{equation}
\label{eq:jel:rho_separation}
\begin{aligned}
    \rho_i (\vec{r}) & = \rho_i^S (\vec{r}) + \rho_i^L (\vec{r}), \\
    \rho_i^S (\vec{r}) & = q_i \delta(\vec{r} - \vec{r_i}) - q_i G_\sigma (\vec{r} - \vec{r_i}), \\
    \rho_i^L (\vec{r}) & = q_i G_\sigma (\vec{r} - \vec{r_i}),
\end{aligned}
\end{equation}
где введено гауссово размытие заряда
\begin{equation}
    \label{eq:jel:gauss_blur}
    G_\sigma (\vec{r}) = \frac{1}{(2\pi \sigma^2)^{3/2}} \exp \left[ - \cfrac{|\vec{r}|^2}{2\sigma^2} \right].
\end{equation}
Дисперсия здесь будет играть роль некоторого параметра.

Выражение для энергии \eqref{eq:jel:coloumb_energy} с учетом \eqref{eq:jel:rho_separation} перепишется в виде:
\begin{equation}
    \label{eq:jel:E_separation}
    E = \frac{1}{2} \sum_{i=1}^N q_i \phi_{-i}^S (\vec{r_i}) + \frac{1}{2} \sum_{i=1}^N q_i \phi^L (\vec{r_i}) - \frac{1}{2} \sum_{i=1}^N q_i \phi_i^L (\vec{r_i})
    = E^S + E^L - E^{self}.
\end{equation}

Уравнение Пуассона для случая, когда заряд распределен по функции Гаусса, может быть решено аналитически. Само уравнение имеет вид:
\begin{equation}
    \label{eq:jel:phi_poisson}
    \Delta \phi_i (\vec{r}) = - \frac{4\pi}{\epsilon}\, G_\sigma (\vec{r}),
\end{equation}
а его решение выражается через функцию ошибок $\mathrm{erf} (x) = \frac{2}{\sqrt{\pi}} \int_0^x e^{-t^2} dt$ следующим образом:
\begin{equation}
    \label{eq:jel:phi_solution}
    \phi_\sigma (r) = \frac{1}{\epsilon r}\, \mathrm{erf} \left( \frac{r}{\sqrt{2}\sigma} \right).
\end{equation}

Таким образом,
\begin{equation}
\label{eq:jel:phi_S_solution}
\begin{aligned}
    \phi_i^S (\vec{r}) &= \frac{1}{\epsilon} \frac{q_i}{|\vec{r} -\vec{r_i}|}\, \mathrm{erfc} \left[ \frac{|\vec{r} - \vec{r_i}|}{\sqrt{2} \sigma} \right], \\
    \phi_i^L (\vec{r}) &= \frac{1}{\epsilon} \frac{q_i}{|\vec{r} -\vec{r_i}|}\, \mathrm{erf} \left[ \frac{|\vec{r} - \vec{r_i}|}{\sqrt{2} \sigma} \right],
\end{aligned}
\end{equation}
где $\mathrm{erfc} (x) = 1 - \mathrm{erf} (x)$. Теперь, с учетом размытия, можно переписать $\phi^S_{-i} (\vec{r})$ в виде:
\begin{equation*}
    \phi_i^S (\vec{r}) = \frac{1}{\epsilon} \sum_\vec{n} \sum_{j=1}^N {}^\prime \frac{q_j}{| \vec{r} - \vec{r_j} + \vec{n}L |}\, \mathrm{erfc} \left( \frac{|\vec{r} - \vec{r_j} + \vec{n}L}{\sqrt{2} \sigma} \right),
\end{equation*}
а энергию:
\begin{equation*}
    E^S = \frac{1}{2} \sum_{i=1}^N q_i \phi_{-i}^S (\vec{r_i})
    = \frac{1}{2\epsilon} \sum_\vec{n} \sum_{i=1}^N \sum_{j=1}^N {}^\prime \frac{q_i q_j}{|\vec{r_i} - \vec{r_j} + \vec{n}L|}\, \mathrm{erfc} \left( \frac{|\vec{r_i} - \vec{r_j} + \vec{n}L|}{\sqrt{2}\sigma} \right).
\end{equation*}
Это выражение аналогично исходному лишь с той разницей, что сумма обрезается функцией $\mathrm{erfc}$, что упрощает расчет.

Используя $\mathrm{erfc} (x) \xrightarrow[x \rightarrow 0]{} \frac{2}{\sqrt{\pi}} x$, можно получить для $E^{self}$:
\begin{equation}
    \label{eq:jel:E_self_final}
    E^{self} = \frac{1}{\epsilon} \frac{1}{\sqrt{2} \sigma} \sum_{i=1}^N q_i^2.
\end{equation}

Таким образом, осталось посчитать третью, дальнодействующую, часть.
Как было отмечено в разделе \ref{sec:ewald}, дальнодействующая часть считается в обратном пространстве.

Записывая плотность заряда в виде суммы по всем зарядам системы, периодически повторяющимся в пространстве:
\begin{equation*}
    \rho^L (\vec{r}) = \sum_\vec{n} \sum_{j=1}^N q_j G_\sigma (\vec{r} - \vec{r_j} + \vec{n}L),
\end{equation*}
имеем для Фурье-образа:
\begin{equation*}
    \tilde{\rho}\,^L (\vec{k}) = \int_V \sum_{j=1}^N q_j G_\sigma(\vec{r} - \vec{r_j} + \vec{n}L) e^{-i\vec{k}\vec{r}} d \vec{r}
    = \sum_{j=1}^N q_j e^{-i \vec{k} \vec{r_j}} e^{-\sigma^2 k^2 / 2} d\vec{r}.
\end{equation*}
Здесь использован тот факт, что $k$ есть вектор обратной решетки и потому $\exp(-i \vec{k} \vec{n} L) = 1$.

Тогда для Фурье-образа потенциала имеем:
\begin{equation}
    \label{eq:jel:phi_L_fourier}
    \tilde{\phi}\,^L (\vec{k}) = \frac{4\pi}{\epsilon} \sum_{j=1}^N q_j e^{-i \vec{k} \vec{r_j}}\, \frac{ e^{-\sigma^2 k^2 / 2} }{ k^2 },
\end{equation}
а сам потенциал (через обратное преобразование):
\begin{equation}
    \label{eq:jel:phi_L_spatial}
    \phi^L (\vec{r}) = \frac{4\pi}{V \epsilon} \sum_{\vec{k} \neq 0} \sum_{j=1}^N \frac{q_j}{k^2}\, e^{i \vec{k} (\vec{r} - \vec{r_j})} e^{-\sigma^2 k^2 / 2}.
\end{equation}
Вклад члена с $\vec{k} = 0$ нулевой, так как полный заряд примитивной ячейки предполагается нулевым.
Для энергии $E^L$ получаем:
\begin{equation}
    \label{eq:jel:E_L_final}
    E^L = \frac{4\pi}{2V \epsilon} \sum_{\vec{k} \neq 0} \sum_{i=1}^N \sum_{j=1}^N \frac{q_i q_j}{k^2} e^{i \vec{k} (\vec{r_i} - \vec{r_j})} e^{-\sigma^2 k^2 / 2}.
\end{equation}

Определяя структурный фактор $S (\vec{k}) = \sum_{i=1}^N q_i e^{i\vec{k} \vec{r_i}}$, дальнодействующую часть энергии возможно переписать в виде:
\begin{equation}
    \label{eq:jel:E_L_from_S}
    E^L = \frac{4\pi}{2V \epsilon} \sum_{\vec{k} \neq 0} \frac{e^{-\sigma^2 k^2 /2}}{k^2} | S (\vec{k}) |^2
\end{equation}
и получить окончательное выражение для полной энергии:
\begin{equation}
\label{eq:jel:total_energy_final}
\begin{aligned}
    E &= E^S + E^L - E^{self} = \\
    &= \frac{1}{2 \epsilon} \sum_{\vec{n}} \sum_{i=1}^N \sum_{j=1}^N {}^\prime \frac{q_i q_j}{|\vec{r_i} - \vec{r_j} + \vec{n}L|}\, \mathrm{erfc} \left( \frac{ |\vec{r_i} - \vec{r_j} + \vec{n}L| }{ \sqrt{2} \sigma } \right) + \\
    &+ \frac{4\pi}{2V \epsilon} \sum_{\vec{k} \neq 0} \frac{e^{-\sigma^2 k^2 / 2}}{k^2} | S (\vec{k}) |^2 - \frac{1}{\epsilon} \frac{1}{\sqrt{2\pi} \sigma} \sum_{i=1}^N q_i^2.
\end{aligned}
\end{equation}

Собирая результаты воедино, получим гамильтониан модели \acrshort{veg}:
% TODO: (a.kozharin) Подумать, как собрать эту солянку из трех статей в один гамильтониан <- Mon Jun 22 00:30:13 2020
\begin{multline}
    \label{eq:jel:ham_final}
        \hat{H} = \sum\limits_{i=1}^{N} \cfrac{\hat{\vec{p}}_{i}^2}{2m}\, +
        \frac{1}{2 \epsilon} \sum_{\vec{n}} \sum_{i=1}^N \sum_{j=1}^N {}^\prime \frac{q_i q_j}{|\hat{\vec{r}}_i - \hat{\vec{r}}_j + \hat{\vec{n}}L|}\, \mathrm{erfc} \left( \frac{ |\hat{\vec{r}}_i - \hat{\vec{r}}_j + \hat{\vec{n}}L| }{ \sqrt{2} \sigma } \right) + {} \\
        {} + \frac{4\pi}{2V \epsilon} \sum_{\vec{k} \neq 0} \frac{e^{-\sigma^2 k^2 / 2}}{k^2} | \hat{S} (\vec{k}) |^2 - \frac{1}{\epsilon} \frac{1}{\sqrt{2\pi} \sigma} \sum_{i=1}^N q_i^2.
\end{multline}
Здесь $\hat{S} (\vec{k}) = \sum\limits_{i=1}^{N} q_{i} e^{i \vec{k} \hat{\vec{r}}_{i}}$, а оператор $\hat{\vec{n}}L$ соответствует координатному сдвигу на целое число ячеек.

\section{Усреднение по направлениям и изотропная модель \texorpdfstring{\acrshort{veg}}{ВЭГ}}
Непосредственное суммирование по Эвальду в том виде, который приведен в секции \ref{sec:ewald}, имеет два практических недостатка:
\begin{enumerate}
    \item Вычислительная сложность (нагрузка на процессор компьютера) быстро растет с ростом числа частиц в главной ячейке (как $N^2$).
    \item Комбинирование дальнодействующего кулоновского взаимодействия с периодическими граничными условиями может привести к появлению неизотропического электрического поля, имеющего кубическую симметрию в кристаллической решетке, составленной из главных ячеек.
        Поэтому любая процедура суммирования кулоновских сил в неупорядоченной системе может считаться подходящей лишь в том случае, если погрешность результата, вызванная этой искуственной симметрией, пренебрежимо мала.
\end{enumerate}
В \cite{jel:pre-averaged_summation} предлагается новый метод суммирования, основанный на предварительном усреднении вдоль всех направлений главной ячейки.
Предполагается, что ячейка имеет вид куба с объемом $V = L^3$ и на систему наложены периодические граничные условия вместе с условиями электронейтральности.

Для начала заметим, что подбором параметра размытия $\sigma$ можно добиться возможности отбросить значительную часть членов суммирования в \eqref{eq:jel:total_energy_final} \cite{rapaport:terms_cutoff}. 
В дальнейшем будем предполагать, что используемый параметр размытия позволяет  осуществить такое упрощение.

Приведем \eqref{eq:jel:total_energy_final} к виду, полученному в оригинальной работе Эвальда \cite{ewald:summation_original}.
Для этого необходимо сделать замену $\delta = \frac{\sqrt{2} \pi}{\sigma}$, а $| S (\vec{k}) |^2$ явно раскрыть с выделением тригонометрических функций. Члены порядка $\exp(-\delta^2 L^2)$ будут давать малый вклад в энергию, если выбрать типичное значение $\delta = 5 / L$ \cite{rapaport:terms_cutoff}.
Рассматривая энергию кулоновского взаимодействия $N$ ионов в главной ячейке, получим:
\begin{equation}
    \label{eq:mean:E_coloumb_main-cell}
    E_N = \sum\limits_{i=1}^{N} q_i \phi (\vec{r}_{i}),
\end{equation}
где $\phi (\vec{r}_{i})$ ---~ электростатический потенциал $i$-го иона в точке $\vec{r}_{i}$.
% TODO: (a.kozharin) Вот тут мы просто прыгаем к виду Эвальда. Нет, выкладки стоит самому провести и проверить, если останется время. Но, мне кажется, если считать только энергию главной ячейки и уйти на обозначения Эвальда, то должно быть все ок <- Mon Jun 22 23:53:31 2020
Согласно \cite{ewald:summation_original}, $\phi$ имеет вид:
\begin{equation}
    \label{eq:mean:phi_as_sum}
    \phi (\vec{r}_{i}) = \phi_1 (\vec{r}_{i}) + \frac{1}{2}\, \sum\limits_{j \neq i}^{N} \phi_2 (\vec{r}_{i}, \vec{r}_{j}),
\end{equation}
где в отстутствии внешнего поля $\phi_1$ есть константа:
\begin{equation}
    \label{eq:mean:phi_1}
    \phi_1 = \cfrac{q_{i}}{L}\, \left( \frac{1}{2\pi}\, \sum\limits_{n > 0}^{} \frac{1}{n^2}\, e^{-\pi^2 n^2 / \delta^2} - \cfrac{\delta}{\sqrt{\pi}}\, \right),
\end{equation}
а парный потенциал получается следующим:
\begin{equation}
    \label{eq:mean:phi_2}
    \phi_2 = q_{j} \left[ \frac{1}{r_{ij}}\, \mathrm{erfc} \left( \delta \cfrac{r_{ij}}{L}\, \right) + \frac{1}{2\pi L}\, \sum\limits_{\vec{n} > 0}^{} \frac{1}{n^2}\, e^{-\pi^2 n^2 / \delta^2} \cos \left( \frac{2\pi}{L}\, \vec{n} \cdot \vec{r}_{ij} \right) \right].
\end{equation}
Здесь $\vec{n} / L$ есть трехмерный взаимный вектор решетки ($n = | \vec{n} |$), $\vec{r}_{ij} = \vec{r}_{i} - \vec{r}_{j}$, $r_{ij} = |\vec{r}_{ij}|$.
Параметр $\delta / L$ обычно называют параметром Эвальда.

Учитывая, что все направления главной решетки в изотропической жидкости должны быть равноправны, обе части \eqref{eq:mean:phi_2} можно усреднить вдоль всех всех направлений вектора $\vec{n}$.
Обозначая $\phi_2 (r_{ij}) \equiv \langle \phi_2 (\vec{r}_{ij}) \rangle$, получим:
\begin{equation}
    \label{eq:mean:phi_2_averaged}
    \phi_2 (r_{ij}) = \frac{q_{j}}{r_{ij}} \left[ \mathrm{erfc} \left( \delta \cfrac{r_{ij}}{L}\, \right) + \frac{1}{2\pi^2}\, \sum\limits_{\vec{n} > 0}^{} \frac{1}{n^3}\, e^{-\pi^2 n^2 / \delta^2} \sin \left( \frac{2\pi}{L}\, n r_{ij} \right) \right].
\end{equation}
Поскольку $\mathrm{erfc}(x) - 1$ и $\sin (x)$ являются нечетными функциями, возможно следующее разложение в ряд:
\begin{equation}
    \label{eq:mean:phi_2_series}
    \phi_2 (r_{ij}) = \cfrac{q_{j}}{r_{ij}}\, \left( 1 + \sum\limits_{k \geq 0}^{} C_{k} r_{ij}^{2k+1} \right),
\end{equation}
где коэффициенты $C_k$ можно найти прямым разложением в ряд Маклорена \eqref{eq:mean:phi_2_averaged} с использованием формулы Эйлера-Маклорена, обобщенной на случай суммирования по трехмерному пространству.
Это в результате даст следующие выражения для членов ряда \cite{jel:pre-averaged_summation}:
\begin{equation}
    \label{eq:mean:C_coeffs}
    \begin{aligned}
        C_0 &= \frac{1}{\pi}\, \sum\limits_{\vec{n} > 0}^{} \frac{1}{n^2}\, e^{-\pi^2 n^2 / \delta^2} - \frac{2\delta}{\sqrt{\pi}}, \\
        C_1 &= \frac{2\pi}{3 L^3}, \\
        C_k &= 0, \quad k > 1.
    \end{aligned}
\end{equation}
Если теперь учесть условие электронейтральности, то окажется, что член в \eqref{eq:mean:phi_2_series}, не зависящий от расстояния (пропорциональный $C_0$), уничтожает вклад от $\phi_1$ в \eqref{eq:mean:phi_as_sum}.
Это означает, что суммарная энергия кулоновского взаимодействия в главной ячейке может быть описана суммой $E_N = \sum_{i=1}^N \phi (r_{ij})$, где
\begin{equation}
    \label{eq:mean:phi_final}
    \phi (r_{ij}) = \frac{q_i q_j}{r_{ij}}\, \left[ 1 + \frac{1}{2}\, \left( \frac{r_{ij}}{r_{m}} \right)^3 \right]
\end{equation}
и $r_{m} = (3 / 4 \pi)^{1 / 3} L$ ---~ радиус эквивалентной по объему сферы главной ячейки: $\frac{4}{3} \pi r_{m}^3 = L^3$.

Результат \eqref{eq:mean:phi_final} и есть усредненный по направлениям потенциал.
Он обладает следующими свойствами:
\begin{enumerate}
    \item Он стремится к чистому кулоновскому парному потенциалу при малых межионных расстояниях;
    \item Его первая производная равна нулю в $r_{m}$;
    \item Его значение в минимуме $r = r_m$ отлично от нуля и равно $\phi (r_m ) = 3 q_i q_j / 2 r_m$.
\end{enumerate}

Значение потенциала в точке $r_m$ можно обнулить, прибавив к нему постоянную добавку $-\phi (r_m)$.
Воспользовавшись условием электронейтральности, получаем окончательное выражение суммарной кулоновской энергии главной ячейки в следующем виде:
\begin{equation}
    \label{eq:mean:E_coloumb_main-cell_final}
    E_N = - \sum\limits_{i=1}^{N} \frac{3q_{i}^2}{4\pi r_m}\, + \frac{1}{2}\, \sum\limits_{i=1}^{N} \sum\limits_{j=1, j \neq i}^{N} \tilde{\phi} (r_{ij}),
\end{equation}
где
\begin{equation}
    \label{eq:mean:phi_tilde}
    \tilde{\phi} (r) =
    \begin{cases}
        \cfrac{q_i q_j}{r}\, \left\{ 1 + \cfrac{1}{2}\, \left( \cfrac{r}{r_m}\,  \right) \left[ \left(\cfrac{r}{r_m}\,   \right)^2 - 3 \right] \right\}, \quad r < r_m \\
        0, \quad r \geq r_m
    \end{cases}
\end{equation}
есть эффективный короткодействующий потенциал межчастичного взаимодействия, который равен нулю вместе со своими производными в точке $r_m$ и остается таковым при $r > r_m$.

Объединяя полученное выражение с кинетической энергией, получим \textit{гамильтониан изотропной модели \acrshort{veg}} в координатном представлении:
% Там жесткая привязка на сравнение с r_m и я не придумал, как ее в общий вид впихнуть. Поэтому пока только в координатном представлении.
\begin{equation}
    \label{eq:mean:ham_final}
    \hat{H} = \sum\limits_{i=1}^{N} \frac{\hat{\vec{p}}_{i}^2}{2m}\, - \sum\limits_{i=1}^{N} \frac{3q_{i}^2}{4\pi \hat{r}_m}\, + \frac{1}{2}\, \sum\limits_{i=1}^{N} \sum\limits_{j=1, j \neq i}^{N} \tilde{\phi} (\hat{r}_{ij}).
\end{equation}
